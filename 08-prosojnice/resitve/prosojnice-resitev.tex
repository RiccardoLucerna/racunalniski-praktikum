\documentclass{beamer}
\usepackage{array}     % določili <{...} in >{...} pri tabelah
\usepackage{tikz}
\usepackage{pgfplots}
\usepgfplotslibrary{external}
\usetikzlibrary{math}
\usepackage{predavanja}

% definicije in izreki
\newtheorem{definicija}{Definicija}
\newtheorem{izrek}{Izrek}

\begin{document}

\title{Matematični izrazi in uporaba paketa \texttt{beamer}}
\subtitle{\emph{Matematičnih} nalog ni treba reševati!}
\institute[FMF]{Fakulteta za matematiko in fiziko}
\date{}

% Naslovna stran
\frame{\titlepage}

\begin{frame}
	% Z ukazom frametitle določite naslov prosojnice
	\frametitle{Kratek pregled}
	% Določite, da naj se kazalo vsebine odkriva postopoma.
	% Ker ni videti preveč lepo, pomožni parameter zakomentirajte.
	\tableofcontents %[pausesections]
 \end{frame}

% Kje v PDF datoteki se pojavi naslov razdelka?
\section{Paket \texttt{beamer}}
\input{prosojnice-resitve/1-paket-beamer.tex}

\section{Paketa \texttt{amsmath} in \texttt{amsfonts}}
\input{prosojnice-resitve/2-paketa-amsmath-amsfonts.tex}

\section[Matematika, 1. del\\\large{Analiza, logika, množice}]{Matematika, 1. del}
\input{prosojnice-resitve/3-analiza-logika-mnozice.tex}

\section{Stolpci in slike}
\input{prosojnice-resitve/4-stolpci-slike.tex}

\section{Paket \texttt{beamer} in tabele}
\input{prosojnice-resitve/5-beamer-tabele.tex}

\section[Matematika, 2. del\\\large{Zaporedja, algebra, grupe}]{Matematika, 2. del}
\input{prosojnice-resitve/6-zaporedja-algebra-grupe.tex}

\end{document}